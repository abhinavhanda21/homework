\documentclass[12pt]{homework}
 
\usepackage[margin=1in]{geometry} 
\usepackage{amsmath,amsthm,amssymb}

\newcommand{\hwname}{Abhinav Handa}
\newcommand{\hwemail}{1411227}
\newcommand{\hwtype}{Homework}
%\newcommand{\hwclass}{ICTA}
%\newcommand{\hwlecture}{0}
%\newcommand{\hwsection}{Z}

% This is just used to generate filler content. You don't need it in an actual
% homework!
\usepackage{lipsum}

\begin{document}
\maketitle

\section{Question}
  What is the need of Information and communication technology in rural development and agriculture?

 

%  This question's number will be auto-incremented.

  %\lipsum[2]
\textbf{Answer}
Information and Communication Technology abbreviated as ICT consist of Information technology, enterprise software, audio-visual system, middleware using which user can access, store, transmit and modify information as required. Exponential growth of internet user, invention of modern communication devices, significant development in cloud and grid computing etc. have helped ICT to flourish as an rapid developed technological field in the last decade. \\

\subsection{Role of ICT in Rural Development}

 India is a country of villages and about half of the villages have very poor socio-economic
conditions. Rural development is an integrated concept of growth, and poverty
elimination has been of paramount concern in all the five year plans. Rural development is a systematic ongoing process of improving the quality of life by socioeconomic well being of the people living in rural areas. Rural Development (RD)
programmes comprise of following:
\begin{itemize}
\item Provision of basic infrastructure facilities in the rural areas e.g. schools, health facilities,
roads, drinking water, electrification etc.\\
\item Improving agricultural productivity in the rural areas.\\
\item Provision of social services like health and education for socio-economic development.\\

\item Implementing schemes for the promotion of rural industry increasing agriculture
productivity, providing rural employment etc.\\
\item Assistance to individual families and Self Help Groups (SHG) living below poverty line by providing productive resources through credit and subsidy.\\
\end{itemize}
The significance of bridging this divide in developing countries stems mainly from the fact that rural areas often lack or lag behind urban areas in terms of essential infrastructure and services such as transportation, health, education and government services. \\


\end{itemize}

\subsubsection{ICT in Education}
In many developing countries bringing a large percentage of students to education system is a great challenge. The reasons may be the geographical location, socio-economic condition etc. As example the north east states of India many villages are scattered in impassable hill regions, West-indies and Filipinos are mainly scattered islands. Poor transport facility discourages the rural students to come to school regularly. Scarcity of efficient teacher in the rural schools and a large student teacher ratio to the student side is also a reason for dropout of a large percentage of students in the midway of their education.\\

\begin{itemize}

\item Thus a great mismatch of education quality is observed when comparison is made with rural and urban students. Adoption of ICT in education can minimize the gap. Role of a teacher is shifted from leader to facilitator in ICT based education system. Adoption of ICT in teaching system enable and support the move from traditional `teacher-centric' teaching styles to more `learner-centric' methods. \\

\item A diverse group of students can learn simultaneously even in the absence of teacher. An online repository must be maintained for accessing the study materials 24x7. \\

\item There must be facility for teleconferencing, video conferencing with experts and for this a certain pre defined time span must be broadcasted to the target learners. A pre assigned interactive session may provide the opportunity to the geographically diverse learners to interact with each other.\\
\end{itemize}

\subsubsection{ ICT in Healthcare}
The medical facility is the mostly neglected section in connection to the rural people. In the perspective of developing countries there is no health center, even not a degree holder doctor available in each village. In many rural hospital there is no full time doctor. Even the doctors do not want to stay in rural areas due to lack of facility, opportunity, poor communication facility etc. For this reason the rural people depend on the quackish even on ojha for health issues. This gives an alarming figure of child death and mother death in rural areas. ICT has a great role to play in health section in rural areas. Adoption of telemedicine in some rural areas of India has given an encouraging result for its accecibility, affordability and availability. With this ICT based facility a small E health kiosk with a trained person can provide medical facility to a large number of people. When a patient is brought to the health kiosk, he enters the health details and problems of the patient to a central server. The server communicates with some doctor in district or urban hospital. The person at the kiosk communicates with the doctor to the other side and performs check up and gives medicines according to the instructions of the doctor.\\



\subsection{Role of ICT in Agriculture}


The agricultural sector in India is currently passing through a difficult phase. India is moving towards an agricultural emergency due to lack of attention, insufficient land reforms, defective land management, non-providing of fair prices to farmers for their crops, inadeuate investment in irrigational and agricultural infrastructure in India, etc. India's food production and productivity is declining while its food consumption is increasing. The position has further been worsened due to use of food grains to meet the demands of bio fuels. #ven the solution of import of food grains would be troublesome, as India does not have ports and logistical systems for large-scale food imports.\\

The application of Information and Communication Technology (ICT) in agriculture is increasingly important. E-agriculture is an emerging field focusing on the enhancement of agricultural and rural development through improved information and communication processes. More specifically, e-)griculture involves the conceptualization, design, development, evaluation and application of innovative ways to use information and communication technologies (ICT) in the rural domain, with a  primary focus on agriculture. )ll stakeholders of agriculture production system need information and knowledge about these phases to manage them efficiently. ICTs are most natural allies to facilitate the outreach of aricultural extension system in the country.\\

\subsubsection{ICTs and Farmer's Advisory Services}
The most widely used and available tools of farmers! advisory services are- telephone based Tele Advisory Services, the mobile based agri advisory services, television and radio based mass media  programmes, web based online agri advisory services, video-conferencing, on-line agri video channel, besides traditional media like, printed literature, newspapers, and farmers exhbition.\\

\subsubsection{ICTs in Animal Disease Management}
The use of ICT in animal husbandry and hospital management dates back to the period of arrival of computers. Since then various ICT tools are used at different levels. Conventional communication modalities like print media, radio broadcastings, television, CD-R0Ms, Handheld computers have been very widely used. Recent concepts like Internet, Geographical Information System (GIS), Global Positioning System (GPS), Database Management, Computer Aided Design (CAD), computer networking, Artificial Intelligence adds strength and efficiency to the ICT in animal disease management. Most of the ICT tools currently used are in 4erd 4ealth management.\\

\newline
\textbf{Conclusion}

The impact of ICT in the rural development of the developing nations are discussed in this paper. The authors have mainly focused on the role of ICT in education, agriculture, healthcare and disaster management of rural area. ICT is an examined key for development of the geographically scattered rural people in developed nation and it is getting its popularity in the developing nations. The primary cost for establishment and set up of ICT infrastructure may be a barrier for developing nation but its enormous usefulness for the rural people can not be denied. Though education, agriculture, healthcare etc. are common to all rural regions, but there are several other sections like tourism, banking and finance etc. in which ICT also has a great role to play.\\

%fair etc.Most of the agricultural institutes and organiations have their own telephone "ased advisory services for farmers which provide telephone "ased )gri advisory services through a dedicated telephone num"er to provide real-time information and advisory. 
  % \begin{induction}
   %%  Here I have my base case.
     %%That doesn't mean it's not important though!
    %\indhyp
     % Assume cool things to make proof work. Look, math:
      %\[a^2 + b^2 = c^2\]
    %\indstep
     %%written in Latin sound more true.  Lorem ipsum dolor sit amet, consectetur
      %adipiscing elit. Maecenas tempor risus in dapibus aliquam. Donec at
      %euismod dui. In libero turpis, blandit quis vestibulum ac, rutrum sit amet
      %est. Suspendisse nec lacus vel dui lobortis lacinia at sit amet risus.
      %Fusce dui ex, imperdiet nec finibus ut, bibendum a lacus.
  %\end{induction}

  %Therefore, we have proven the claim by induction in the \texttt{induction} environment.

%\question
 % Use the arabicparts environment to include the questionCounter number in the list.
  %\begin{arabicparts}
   %%\item ???
    %\item Profit!
  %\end{arabicparts}

  %\lipsum[7]

%\question
 % Use the alphaparts environment to for letters instead of numbers.
  %\begin{alphaparts}
   % \item
    %  Use \LaTeX

     % \lipsum[8]
    %\item ???
    %\item Profit!
 

\end{document}
